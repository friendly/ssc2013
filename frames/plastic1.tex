\begin{frame}[containsverbatim]
	\frametitle{Example: Two-way MANOVA-- Plastic film data}

  \begin{itemize}
  	\item Data from an experiment to deterimne the optimal conditions
	for extruding plastic film.
	\begin{itemize*}
		\item Factors: Rate of extrusion (low/high), amount of
		additive (low/high)
		\item Responses: Tear resistance, film gloss, opacity
		\item $\rightarrow$ $2 \times 2$ MANOVA design, 3 responses, $n=5$ per
		cell.
	\end{itemize*}

	\item HE plots show main effects, interactions and linear hypotheses
	in relation to each other 
  \end{itemize}
\begin{CodeInput}
R> plastic.mod <- lm(cbind(tear, gloss, opacity) ~ 
		rate*additive, data=Plastic)
R> Manova(plastic.mod, test.statistic="Roy")
\end{CodeInput}
\begin{CodeOutput}[fontsize=\small]
Type II MANOVA Tests: Roy test statistic
              Df test stat approx F num Df den Df   Pr(>F)   
rate           1    1.6188   7.5543      3     14 0.003034 **
additive       1    0.9119   4.2556      3     14 0.024745 * 
rate:additive  1    0.2868   1.3385      3     14 0.301782   
---
Signif. codes:  0 '***' 0.001 '**' 0.01 '*' 0.05 '.' 0.1 ' ' 1 
\end{CodeOutput}
	\end{frame}

\begin{frame}
  \frametitle{Example: Two-way MANOVA-- Plastic film data}
  \begin{columns}[T]
    \begin{column}{.4\textwidth}
		\begin{itemize}
			\item<1-> Main effects \& interaction-- evidence scaling
			\item<1-> Only effect of \texttt{rate} exceeds the \E ellipse. Why?
			\item<2-> Main effects \& interaction-- effect scaling
		\end{itemize}
    \end{column}
    \begin{column}{.6\textwidth}
    \includegraphics<1>[width=\textwidth,clip]{fig/plastic1-1a}
    \includegraphics<2>[width=\textwidth,clip]{fig/plastic1-1b}
    \end{column}
  \end{columns}
\end{frame}

\begin{frame}<beamer>
%  \frametitle{Example: Two-way MANOVA-- Plastic film data}
  \frametitle{Testing linear hypotheses}
  \begin{columns}[T]
    \begin{column}{.4\textwidth}
		\begin{itemize}
			\item<1-> Test composite hypotheses:
			\item<1-> \texttt{Group = rate * additive} (3 df)
			\item<2-> \texttt{Main = rate + additive} (2 df)
		\end{itemize}
    \end{column}
    \begin{column}{.6\textwidth}
    \includegraphics<1>[width=\textwidth,clip]{fig/plastic1-2a}
    \includegraphics<2>[width=\textwidth,clip]{fig/plastic1-2b}
    \end{column}
  \end{columns}
\end{frame}

\begin{frame}
  \frametitle{3D HE plots}
%  \framesubtitle{3D HE plots}
  \begin{columns}[T]
    \begin{column}{.4\textwidth}
		\begin{itemize}
			\item<1-> 3D HE plot shows ellipsoids for \H and \E matrices
			\item<1-> 1 df hypotheses \implies lines
			\item<2-> 2 df hypotheses \implies ellipses
			\item<2->\texttt{heplot3d} function provides interactive
			rotation
			\item<2->This view shows the significant main effects of
			\texttt{rate} and \texttt{additive}
		\end{itemize}
    \end{column}
    \begin{column}{.6\textwidth}
    \includegraphics<1>[width=\textwidth,clip]{fig/plastic1-3a}
    \includegraphics<2>[width=\textwidth,clip]{fig/plastic1-3b}
    \end{column}
  \end{columns}
\end{frame}

